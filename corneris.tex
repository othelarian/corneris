\documentclass[11pt,oneside]{book}
\usepackage[utf8]{inputenc}
\usepackage[T1]{fontenc}
\usepackage[a4paper,margin=0.8in]{geometry}

\usepackage[explicit]{titlesec}

\usepackage{fancyhdr}
\usepackage{french}
\usepackage{multicol}


\usepackage{chapheader}


\title{Corneris JDR}
\author{othelarian}
\date{}

\pagestyle{fancy}

\begin{document}

\maketitle

\tableofcontents

\begin{multicols}{2}
    [    
        \chapter{Création}
        \linebreak
        \emph{Et de la poussière du néant naîtra les légendes qui parcouront nos vies de par les rêves qu'elles inspirent.}
    ]

    Vos lunettes sur le nez, une tasse de café fumante près de vous, et peut-être même une couverture sur les genoux et le chat lové sur vos pieds, vous êtes désormais fin prêt.

    Bienvenue chers yeux avides de lecture dans ces humbles pages. Le contenu que vous avez actuellement sous les yeux est celui rassemblé par votre modeste serviteur, pour constituer la pierre angulaire sur laquelle s'appuie la mécanique merveilleuse du jeux de rôle ''Corneris''. Ce socle de règles et principes a été patiemment édifié au fil des parties, lectures, et tests, et reflète bien sûr une vision unique et subjective de la manière dont votre dévoué servant perçoit ce qu'est pour lui ce qui doit être.

    Vous avez donc devant vous un recueil de règles pour pouvoir créer et jouer un groupe de personnages atour d'une table, d'un forum, d'une chaîne de mail, etc, à vous de choisir le support. Ces règles sont issues principalement d'autres jeux (il est si rare de réinventer la roue de nos jours), sorties de leurs contextes respectifs, mélangées pour donner lieu à une nouvelle recette. Ce n'est sans doute pas parfait, ça ne plaira sans doute pas à tout le monde, mais ça fonctionne, donc on s'en fiche un peu. N'oubliez jamais la première règle de tout bon jeux de rôle : si la règle ne fonctionne pas, ce n'est pas grave, le tout est que la partie avance dans la bonne humeur. C'est un jeu, alors amusez-vous avant tout !

    Mais trêve de bavadarge, rentrons sans plus attendre dans le lard du sujet.

    \section{Source d'inspiration}

        Parce qu'il faut savoir rendre à César ce qui appartient à César, voici la liste des principaux jeux qui ont servis à l'élaboration de ce recueil :

        \begin{itemize}
            \item Tri-Stat dX / BESM (facilement reconnaissable vu que toute la feuille de personnage provient plus ou moins de ce système)
            \item Hero System (moins évident, mais au final Tri-Stat et le Hero System partagent tellement de point commun)
            \item d6 System et ces nombreux dérivés (je n'ai jamais aimé la notion de jet avec un maximum à ne pas dépasser, et puis les jets explosifs, c'est le bien)
            \item Iron Kingdoms (Son influence est sans doute celle qui passera le plus inaperçu, mais elle est bien là)
            \item Savage Worlds (les relances, c'est cool)
            \item Arkeos / EWSystem (y goûter c'est avoir toujours ce petit arrière goût de reviens-y, alors j'y reviens toujours)
        \end{itemize}

        Si un jour l'envie vous prend, et que vous n'avez jamais testé l'un d'eux, je vous conseille fortement d'y jouer, juste un peu. Le changement est souvent bénéfique, aussi bien en tant que joueur qu'en tant que maître de jeu. Tentez votre chance !

    \section{La naissance des héros}

        Parce qu'il faut bien commencer quelque part, la première et la plus raisonnable des choses à faire semble d'abord d'ouvrir une ou plusieurs bières\footnote{ou tout alcool approprié}, trouver des sièges ou tout autre support pouvant servir la même fonction, et discuter. Qui ? Le joueur et le maître de jeu, ou tout les joueurs ensemble, ou tout le monde ensemble, peu importe. La vraie question est : de quoi ?

        Car au début vient le héros, ce qu'il est, ce qu'il perçoit, ce qu'il sait faire, ce qui l'aime, ses plus grandes joies comme ses plus grandes peurs. De mémoire de maîtres, aucun début ne ressemble à un autre, chacun est unique, agissant telles les fondations même de l'aventude qui s'annonce. Une phase de discussion est donc nécessaire, pour harmoniser les pensées et accorder les esprits. L'important est que chacun des participants aient une vision commune, à la fois de l'univers, mais aussi de chaque héros prenant part à l'histoire.

        Ici pas de description par le détail, mais plutôt les grosses lignes, comme un brouillon tracé à la va-vite pour ne pas laisser s'enfuir l'idée. Les descriptions les plus complexes à cette étape ne doivent pas excéder la douzaine de lignes sans raison valable et justifiée. N'oubliez pas l'adage : plus c'est précis, plus c'est immobile.

    \section{Les quatre 'C'}

        L'image mentale étant maintenant dessinée, que les mesures commencent !

        Et de fil en aiguille viennent les quatre 'C' :

        \begin{description}
            \item[Caractéristique :] il y a trois caractéristique dans Corneris : l'Âme, le Corps et l'Esprit.
            \item[Capacité :] parce qu'un héros, ça a toujours des capacités hors-norme, il faut bien qu'elles soient quelque part.
            \item[Compétence :] un personnage apprend très souvent à faire tout un tas de choses, d'où les compétences.
            \item[Contrainte :] et bien sûr pour terminer, personne n'est parfait.
        \end{description}

        % TODO : finir la présentation des quatres C
    
    \section{Les caractéristiques}
        
        % TODO : section dédiées au caractéristiques

        % TODO : caractéristiques secondaires :
        % - volonté
        % - agilité
        % - force

    \section{les capacités}

        % TODO : section dédiées aux capacités

        % TODO : ne pas oublier : dans la (les?) capacité concernant un éventuel suivant ou assimilé, bien noter que le suivant est controllé par le joueur, et non par le MJ (SW cool feature)

    \section{Les compétences}

        % TODO : section dédiées aux compétences

        % TODO : rappel : voir IK pour la liste des compétences, plutôt partir sur du générique avec spé

        % TODO : coupler les skills IK avec le système EWS (very cool stuff)

    \section{Les contraintes}

        % TODO : section dédiées aux contraintes

        % TODO : SW bad feature : éviter à tout prix les contraintes qui n'ont pas d'impact direct en terme de jeu, mais uniquement celles qui engendre un effet sur la mécanique de jeu

    \section{Finalisation}

        % TODO : calcul des valeurs secondaires / dérivées
        % - valeur de combat / défense
        % - niveau de vie (et non points de vie)
        % - les points d'éclat

        % TODO : explication sur les touches finales
        
    \section{Pour les maîtres}

        % TODO : note au MJ

        % TODO : sous sections pour les PNJs

        % TODO : sous sections pour les objets magiques ???

    \section{Évolution}

        % TODO : zone où il est question de l'évolution des personnages

        % TODO : rappel : les joueurs gagne de l'xp à chaque fin de scène et non de session



    % ############################
    % FINIR LE TEXTE
    % ############################


    
\end{multicols}

\begin{multicols}{2}
    [
        \chapter{Altération}
        \linebreak
        \emph{Où il est question des mécanismes secrets qui transforment chaque chose, chaque être.}
    ]

    Maintenant que les personnages sont prêts à en découdre, il reste désormais déterminer les moyens mis à leur disposition pour qu'ils puissent traverser les situations périlleuses que place avec amour et dévotion\footnote{deux êtres maléfiques dont il faut absolument se méfier} le vénéré maître de jeu.

    % TODO : finir l'introduction

    \section{Notion de temps}

    % TODO : où il est question de scènes, de tour et d'initiative

    % TODO : bien distinguer la scène de la session ou de la campagne

    \section{Avoir l'initiative}

    % TODO : comment calculer l'init (EWS ou SW, pas encore fait mon choix au final)

    % TODO : modification possible à l'init :
    % - via une capacité
    % - la surprise

    \section{Confrontation}

    % TODO : notion d'action et réaction

    % TODO : le jet -> 2d6
    % - seuil : 7+bonus-malus (voir EWS)

    % TODO : jets explosifs !

    % TODO : les difficultés

    % TODO : les relances (another SW cool feature)

    % TODO : confrontations étendues et combinées

    % TODO : utilisation des points d'éclat

    \section{Environnement dangereux}

    % TODO : règles sur la fatigue, sur l'environnement, etc

    % TODO : principalement, voir SW



    % ############################
    % FINIR LE PLAN
    % ############################



\end{multicols}

\begin{multicols}{2}
    [
        \chapter{Destruction}
        \linebreak
        \emph{Et quand l'onde vient à s'éteindre, après tant et tant de ricochets, le silence alors emplit l'espace, marquant par son écrasante présence la fin d'une bataille, et le début d'une autre.}
    ]

    % TODO : Combat !!!!!!

    Où comment parler simplement et efficacement du combat.

    % TODO : finir l'introduction

    % TODO : fusion de EWS et IK, ça ne va pas être simple


    % ############################
    % FAIRE LE PLAN
    % ############################


\end{multicols}

\end{document}