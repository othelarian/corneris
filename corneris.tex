\documentclass[11pt,oneside]{book}
\usepackage[utf8]{inputenc}
\usepackage[T1]{fontenc}
\usepackage[a4paper,margin=0.8in]{geometry}

\usepackage[explicit]{titlesec}

\usepackage{fancyhdr}
\usepackage{french}
\usepackage{multicol}


\usepackage{chapheader}


\title{Corneris JDR}
\author{othelarian}
\date{}

\pagestyle{fancy}

\begin{document}

\maketitle

\tableofcontents

\begin{multicols}{2}
[    
    \chapter{Création}
    \emph{Et de la poussière du néant naîtra les légendes qui parcouront nos vies de par les rêves qu'elles inspirent.}
]

Bienvenue chers yeux avides de lecture dans ces humbles pages. Le contenu que vous avez actuellement sous les yeux est celui rassemblé par votre modeste serviteur, pour constituer la pierre angulaire sur laquelle s'appuie la mécanique merveilleuse du jeux de rôle ''Corneris''. Ce socle de règles et principes a été patiemment édifié au fil des parties, lectures, et tests, et reflète bien sûr une vision unique et subjective de la manière dont votre dévoué servant perçoit ce qu'est pour lui ce qui doit être.

Mais trêve de bavadarge, rentrons sans plus attendre dans le lard du sujet.

\section{La naissance des héros}

Parce qu'il faut bien commencer quelque part, la première et la plus raisonnable des choses à faire semble d'abord d'ouvrir une ou plusieurs bières\footnote{ou tout alcool approprié}, trouver des sièges ou tout autre support pouvant servir la même fonction, et discuter. Qui ? Le joueur et le maître de jeu, ou tout les joueurs ensemble, ou tout le monde ensemble, peu importe. La vraie question est : de quoi ?

Car au début vient le héros, ce qu'il est, ce qu'il perçoit, ce qu'il sait faire, ce qui l'aime, ses plus grandes joies comme ses plus grandes peurs. De mémoire de maîtres, aucun début ne ressemble à un autre, chacun est unique, agissant telles les fondations même de l'aventude qui s'annonce. Une phase de discussion est donc nécessaire, pour harmoniser les pensées et accorder les esprits. L'important est que chacun des participants aient une vision commune, à la fois de l'univers, mais aussi de chaque héros prenant part à l'histoire.

Ici pas de description par le détail, mais plutôt les grosses lignes, comme un brouillon tracé à la va-vite pour ne pas laisser s'enfuir l'idée. Les descriptions les plus complexes à cette étape ne doivent pas excéder la douzaine de lignes sans raison valable et justifiée. N'oubliez pas l'adage : plus c'est précis, plus c'est immobile.

\section{Les quatre 'C'}

L'image mentale étant maintenant dessinée, que les mesures commencent !

Et de fil en aiguille viennent les quatre 'C' :

\begin{description}
    \item[caractéristique :] % TODO : finir la description
    % TODO : finir la liste
\end{description}

% TODO : qu'est-ce que les quatre 'C' ?

% TODO : les caractéristiques

% TODO : les capacités

% TODO : les compétences

% TODO : les contraintes


\end{multicols}

\begin{multicols}{2}
[
    \chapter{Altération}
    \emph{Où il est question des mécanismes secrets qui transforment chaque chose, chaque être.}
]

Maintenant que les personnages sont prêts à en découdre, il reste désormais déterminer les moyens mis à leur disposition pour qu'ils puissent traverser les situations périlleuses que place avec amour et dévotion\footnote{deux êtres maléfiques dont il faut absolument se méfier} le vénéré maître de jeu.


% TODO : la notion de tour, des jets, des difficultés, etc ?


% TODO : test


\end{multicols}

\begin{multicols}{2}
[
    \chapter{Destruction}
    \emph{Et quand l'onde vient à s'éteindre, après tant et tant de ricochets, le silence alors emplit l'espace, marquant par son écrasante présence la fin d'une bataille, et le début d'une autre.}
]

% TODO : Combat !!!!!!

Où comment parler simplement et efficacement du combat.

\end{multicols}

\end{document}